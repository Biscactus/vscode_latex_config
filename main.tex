\documentclass[a4paper,12pt]{article} % Classe du document
%--------------------- Packages ------------------------

\RequirePackage[french]{babel} %Langue du document
\RequirePackage[utf8]{inputenc} %Caractères spéciaux
\RequirePackage[section]{placeins}%Pour placement de section
\RequirePackage[T1]{fontenc} %Quelques lettres qui sont pas inclus dans UTF-8
\RequirePackage{mathtools} %Paquet pour des équations et symboles mathématiques
\RequirePackage{siunitx} %Pour écrire avec la notation scientifique (Ex.: \num{2e+9})
\RequirePackage{float} %Pour placement d'images
\RequirePackage{graphicx} %Paquet pour insérer des images
\RequirePackage[justification=centering]{caption} %Pour les légendes centralisées
\RequirePackage{subcaption}
\RequirePackage{wallpaper}
\RequirePackage{nomencl}
\makenomenclature
\RequirePackage{fancyhdr}
\pagestyle{fancy}
\fancyheadoffset{1cm}
\setlength{\headheight}{2cm}
\setlength{\parindent}{0cm} 
\RequirePackage{url}
\RequirePackage[hidelinks]{hyperref}%Paquet pour insérer légendes dans des sous-figures comme Figure 1a, 1b
\RequirePackage[left=2.5cm,right=2.5cm,top=2cm,bottom=3.5cm]{geometry} %Configuration de la page


%-------------------- Informations sur le rapport ----------------------

\newcommand{\nom}[1]{\renewcommand{\nom}{#1}}


%------- Créer commande immage centralisée ---------------------------
\newcommand{\insererfigure}[4]{
\begin{figure}[H]
\centering
\includegraphics[height=#2]{#1}
\caption{#3}
\label{fig: #4}
\end{figure}
}

\usepackage{pythontex}%pour utiliser python

\begin{document}

%----------- Informations du rapport ---------

\nom{Prénom NOM}

%----------- Initialisation -------------------
  
\lhead{\nom} %Header Left
%\chead{} %Header center
\rhead{\nouppercase{\today}} %Header right
\rfoot{\thepage} %Footer right
\cfoot{titre} %Footer center
\lfoot{titre secondaire} %Footer left

\begin{titlepage}
    %créer la page de garde ici
     \centering   
        \vfill
        {\large \today\par} %Affichage de la date
    
    \end{titlepage}
\tableofcontents %Créer la table de matières
\newpage 

%------------ Corps du rapport ----------------
\section{Général}
Il est possible de créer différente section qui seront mises à jour automatiquement
dans la table des matières.

%------------- Commandes utiles ----------------

\section{Quelques commandes}

Voici quelques commandes utiles :
\subsection{Image}
%------ Pour insérer et citer une image centralisée -----

\insererfigure{logos/cardBrushEx.jpg}{3cm}{Légende de la figure}{Label de la figure}
% Le premier argument est le chemin pour la photo
% Le deuxième est la hauteur de la photo
% Le troisième la légende
% Le quatrième le label
Ici, je cite l'image \ref{fig: Label de la figure}

\subsection{Équation}
%------- Pour insérer et citer une équation --------------

\begin{equation} \label{eq: exemple}
\Delta c = c \cdot \frac{\Delta g}{g}
\end{equation}

L'équation \ref{eq: exemple} est cité ici. 

\subsection{Variables}
% ------- Pour écrire des variables ----------------------

Pour écrire des variables dans le texte, il suffit de mettre le symbole \$ entre le texte souhaité comme : constante $\rho$.

\subsection{Python}
% ------- Pour utiliser python ----------------------------

\begin{pycode}
x = 1+10
\end{pycode}
       
La valeur peut être récupérée dans le texte : $x = \py{x}$.\\
Si la valeur affichée est ??, il suffit d'ouvrir le terminal et d'exécuter la 
commande suivante puis recompiler le document :\\
pythontex main.tex     

\end{document}
